\documentclass{article}

\usepackage{textcomp}
\usepackage{titling}
\usepackage{titlesec}
\usepackage{hyperref}
\usepackage[margin=1in]{geometry}
\usepackage{fancyhdr}

\pagestyle{fancy}
\fancyhf{}
\renewcommand{\headrulewidth}{0pt}

\titleformat
{\section}        % command
{\flushleft\Large\bfseries}      % formating
{}                % label
{0em} %spacing between number and title
{} %content between number and title
[\titlerule]

\titleformat
{\subsection}
{\flushleft\large\bfseries}
{}
{.25em}
{}

\titleformat
{\subsubsection} %[runin] %no indent
{\flushleft}
{}
{0em}
{}[---]

\renewcommand{\maketitle}
{
\begin{center}
\huge \bfseries \theauthor\\
\vspace{.25em}
\mdseries \normalsize ajn6@pdx.edu --- Software Engineer / Data Scientist
\end{center}
}

\title{R\'esum\'e}
\author{Austen Nelson}

\begin{document}
\maketitle

\section{Work Experience}

\subsection{Self Employed $\;\vert\;2014-2017\;\vert\;$ Math and Science Tutor}
Operated a small personal business where I mostly tutored calculus to
engineering students

\subsection{Portland State University $\;\vert\;2019-2021\;\vert\;$ Technical Course Support Specialist}
Karla Fant teaches a majority of the lower division courses (CS162, CS163, CS202) at PSU\\
Responsibilities included:

\begin{itemize}
\item Facilitate homework recitation and labs
\item Proctor programming proficiency demos and written exams
\item Create course materials
\item Grading and administrative tasks
\end{itemize}

\section{Education}

\subsection{Portland State University --- BCS}
Bachelor's of Computer Science completed in May of 2021 with a 3.8 GPA

\section{Programming Languages}
\begin{itemize}
   \item \textbf{Rust} --- My strongest, most modern, and favorite tool to use. Rust is a innovative and exciting
         project started by Mozilla in 2010. The goals of the project are to create a C++ replacement with a
         focus on performance, safety, and concurrency. These values are mostly achieved by the novel ownership
         system that allows for powerful static analysis of the language. The language has seen tremendous growth
         in the recent years and I would love to keep expanding my Rust knowledge.
      \item \textbf{C++} --- Two years of experience in C++ 98 through school projects and tutoring. I have very limited
         exposure to modern C++ concepts.
      \item \textbf{Python} --- Almost all of my data science and AI/ML projects have been done using python and the popular
         libraries of Numpy, Pandas, Matplotlib, Scikit Learn, and Requests. I have also used Python to set up automatic
         deployment of cloud resources using the GCP via their Deployment Manager API.
      \item \textbf{Latex} --- Quite familiar using Latex to create beautiful documents that can be tracked and collaborated easily
         with a VCS like git.
      \item Limited exposure to Java, R, Javascript, HTML/CSS, Bootstrap, Haskell, C, and Bash.
\end{itemize}

\section{Tools}
\begin{itemize}
   \item \textbf{Google Cloud Platform (GCP)} --- My capstone project for graduation was to design a codelab to teach various cloud security
      and vulnerability topics using the GCP. Concepts I am familiar with include IAM, service accounts, compute and storage resources,
      deployment management and automation, and least privilege. See projects section for more information.
   \item \textbf{Git} --- Confident using version control in a team environment. Proficient in branching, pull requests, and merge conflict resolution.
   \item \textbf{SQL} --- Basic understand of SQL concepts and techniques. Most experience working with the Postgres software.
   \item \textbf{Agile} --- Familiar with the general agile concepts and how they can be used in software developed. Used Agile techniques
      in group projects such as capstone.
   \item \textbf{Linux} --- Use Linux as my primary development environment. Familiar with Bash as well as basic system administration concepts
      such as init systems (systemd), users and groups, permissions, and file systems. Vim >>> Emacs
   \item \textbf{Docker} --- Use docker compose for projects and personal projects to simplify build processes.
\end{itemize}

\section{Concepts}
\begin{itemize}
      \item Artificial Intelligence and Machine Learning --- Strong background in applied statistics and linear algebra. Familiar with
         statistical techniques such as naive Bayes, regression analysis, and descriptive analysis. Decent understanding of different
         types of multi-layer perceptrons (CNN, RNN, LSTM, GAN, etc.) and gradient decent for supervised learning.
         See projects section for an example of a 2 layer
         fully connected perceptron classifier I implemented from scratch to do MNIST classification in my college machine learning course.
         Familiar with unsupervised learning techniques such as k-means clustering and modularity based graph algorithms for community
         detection in large networks. See projects section for a graph focused research project where I identified communities of cooking
         ingredients using modularity to generate novel recipes.
      \item Security, Malware, and Hacking --- Took two classes at university focused on computer security. One was an introduction to malware
         where the primary exercises were using disassembly tools (IDA) to solve CTFs. The second was a web and cloud course where we
         completed exercises to demonstrate improper authentication techniques, brute force and fuzzing, improper access controls,
         server-side request forgeries, cross site scripting, injection attacks (SQL, XML, Bash, etc.), improper CORS, cross site request
         forgeries, clickjacking techniques, improper deserialization, and improper cloud configuration. I have also competed in a couple
         different CTF competitions and would love to do more!
      \item Systems Programming --- As a Rust evangelist I have an interest in systems level programming such as data encoding/processing,
         operating systems, embedded systems, and networking. I am currently working on a fullstack web application project where the API
         is implemented using the warp async/await Rust library implemented on top of the Tokio runtime with a Postgres database (see projects).
         I enjoy working with and designing backend systems and APIs to create scalable applications.
\end{itemize}

\section{Personal Interests and Projects}

\subsection{Undergraduate Research Mentorship Program: Culinary Computation}

\subsection{Monte Carlo Simulation of Set\tiny\mdseries\textregistered}
I am interested in the probabilities of this game.\\
Motivations and an analysis I did can be found here:
\href{https://www.github.com/aujxn/set_game_simulator}{Set Simulator}

\subsection{Linear Algebra}
I am co-developed an open source linear algebra library with the help
of Waylon Cude when I was in an advanced linear algebra course at PSU.
The library was to demonstrate my familiarity with sparse matrix operations and algorithms.
As a final project for the class I reimplemented an iterative solver known as GMRES.\\
Here is a link to the project: \href{https://www.gitlab.com/AustenN/matrixlab}{matrixlab}

\subsection{Thunder CTF: Cloud Audit and Incident Response}

\subsection{Primitive Guitar Pedal}

\subsection{Multi-Layer Perceptron}

\subsection{Gardens App}

\section{References}
\begin{itemize}
\item Karla Fant --- Senior Instructor, PSU MCECS --- (503) 725-5394
\item Panayot S Vassilevski --- Computational Research Mathematician, Lawrence Livermoor National Labs
\end{itemize}

\end{document}

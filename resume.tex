\documentclass{article}

\usepackage{textcomp}
\usepackage{titling}
\usepackage{titlesec}
\usepackage{hyperref}
\usepackage[margin=.75in]{geometry}
\usepackage{fancyhdr}

\pagestyle{fancy}
\fancyhf{}
\renewcommand{\headrulewidth}{0pt}

\titleformat
{\section}        % command
{\flushleft\Large\bfseries}      % formating
{}                % label
{0em} %spacing between number and title
{} %content between number and title
[\titlerule]

\titleformat
{\subsection}
{\flushleft\large\bfseries}
{}
{.25em}
{}

\titleformat
{\subsubsection} %[runin] %no indent
{\flushleft}
{}
{0em}
{}[---]

\renewcommand{\maketitle}
{
\begin{center}
\huge \bfseries \theauthor\\
\vspace{.25em}
   \mdseries \normalsize ajn6@pdx.edu --- (971) 806-9229 --- \href{aujxn.dev}{aujxn.dev} --- Software Engineer / Data Scientist
\end{center}
}

\title{R\'esum\'e}
\author{Austen Nelson}

\begin{document}
\maketitle

\section{Education}

\subsection{Portland State University --- BCS}
Bachelor of Computer Science --- 2018-2021 --- Magna cum laude, 3.8

\section{Experience}

\subsection{Self Employed $\;\vert\;2014-2017\;\vert\;$ Math and Science Tutor}
Tutored calculus to engineering students.

\subsection{Portland State University $\;\vert\;2019-2021\;\vert\;$ Technical Course Support Specialist}
Karla Fant teaches the core lower division computer science courses (CS162, CS163, CS202)\\

\begin{itemize}
\item Facilitated homework recitation and labs by presenting weekly materials and answering questions
\item Worked the tutor desk where I debuged student homework code and helped prepare for exams
\item Proctored programming proficiency demos and written exams
\item Created course materials
\item Helped with grading and administrative tasks
\end{itemize}

\subsection{Undergraduate Research Mentorship Program $\;\vert\;2019-2020\;\vert\;$ \href{https://scholar.google.com/citations?user=GhpkHDAAAAAJ&hl=en}{Panayot Vassilevski}}
Received a 4 term academic stipend to pursue independent research with the help of a faculty advisor.
My research project worked with recipe data to create ingredient networks. 
These networks were analyzed as communities using modularity algorithms. Using this community data,
we developed and implemented and algorithm to create original recipes based on certain constraints.
The project involved meeting with my adviser, Panayot, weekly and giving write-ups or presentation to the program manager.
(Project Links: \href{https://github.com/aujxn/research/blob/master/paper/Cullinary_Computation_paper.pdf}{Paper}, \href{https://github.com/aujxn/recipe_analysis}{Code})
(Panayot Links: \href{https://scholar.google.com/citations?user=GhpkHDAAAAAJ&hl=en}{Google Scholar}, \href{http://web.pdx.edu/~panayot/}{PSU}, \href{https://people.llnl.gov/vassilevski1}{LLNL})

\subsection{Capstone Project Team Lead $2021$}
Selected by university staff to lead a team for the capstone graduation requirement. Our
capstone project was to develop university level codelab exercises to teach concepts about cloud security
and incident management through topics like IAM policies and iterative log querying.
(\href{link}{presentation video}, \href{link}{project plan}, \href{https://github.com/NicholasSpringer/thunder-ctf/tree/master/core/levels/defender}{git repo}, \href{link}{website})

\subsection{Monte Carlo Simulation of Card Game Statistics $2019$}
Program that performs a MCS for a few complex statistics about special scenarios in the game Set. Set is a simple card game that demonstrates neat
and complex emergence. Included in the \href{https://github.com/aujxn/set_game_simulator}{repository} is a full written analysis of the results.

\section{Skills}
Rust, SQL, C, C++ 98, Python, Git, Linux, Docker, GCP, AWS, AI/ML, Statistics, Linear Algebra, Sparse Matrices

\end{document}
